\documentclass[UTF8]{ctexart}
\usepackage{geometry}
\geometry{a4paper, margin=2.5cm}
\usepackage{hyperref}
\usepackage{enumitem}
\usepackage{fancyhdr}
\pagestyle{fancy}
\fancyhf{}
\fancyhead[C]{第三次作业报告}
\fancyfoot[C]{\thepage}

\title{第三次作业报告}
\author{刘叙杨 \quad 学号:2023010838\\ \texttt{liuxuyan23@mails.tsinghua.edu.cn}}
\date{}

\begin{document}

\maketitle

\section{实现思路}

本次作业在第二次作业的基础上,增加了流量统计功能、时间流动和用户名密码储存,实现了用户流量使用情况的动态跟踪和持久化存储。

在页面外观上,保留了前两次作业的页面样式。在技术实现方面,主要使用JavaScript完成核心逻辑功能,并利用localStorage实现数据的持久化存储。

\textbf{登录页面逻辑实现}:
\begin{itemize}
    
    \item \textbf{用户认证与注册}:在登录按钮点击事件中,通过 \texttt{querySelector} 获取用户名和密码输入框的值,检查是否为空。如果用户已注册,验证密码正确性;如果是新用户,自动注册并保存用户信息。验证通过后,使用 \texttt{localStorage.setItem('currentUsername', username)} 存储当前用户名,并通过 \texttt{window.location.href} 跳转到成功页面。
    
    \item \textbf{用户名密码储存功能}:实现了完整的用户管理系统,包括:
    \begin{itemize}
        \item \textbf{用户数据存储}:使用 \texttt{localStorage.setItem('registeredUsers', JSON.stringify(registered))} 将用户注册信息以JSON格式存储,数据结构为 \texttt{\{ "用户名": "密码" \}}。
        \item \textbf{用户数据读取}:通过 \texttt{JSON.parse(localStorage.getItem('registeredUsers') || '{}')} 读取已注册用户信息,如果不存在则初始化为空对象。
        \item \textbf{用户验证逻辑}:检查用户名是否存在于注册列表中,如果存在则验证密码匹配性,如果不存在则自动注册新用户。
        \item \textbf{数据持久化}:用户注册信息在浏览器关闭后仍然保留,确保用户无需重复注册。
    \end{itemize}
    
\end{itemize}

\textbf{登录成功页面逻辑实现}:
\begin{itemize}
    \item \textbf{用户名显示}:页面加载时通过 \texttt{DOMContentLoaded} 事件监听器,使用 \texttt{localStorage.getItem('currentUsername')} 获取存储的用户名,并更新到页面显示。
    
    \item \textbf{时间流动功能}:实现了实时在线时长统计系统,包括:
    \begin{itemize}
        \item \textbf{时间初始化}:页面加载时记录开始时间 \texttt{startTime = Date.now()},用于计算在线时长。
        \item \textbf{时间计算}:通过 \texttt{computeTime()} 函数计算从页面加载到当前的时间差,将毫秒转换为小时、分钟、秒数。
        \item \textbf{时间格式化}:使用 \texttt{padStart(2, '0')} 确保时间显示为两位数格式(HH:MM:SS),如"01:23:45"。
        \item \textbf{实时更新}:通过 \texttt{setInterval(computeTime, 1000)} 每秒更新一次时间显示,确保在线时长的实时性。
        \item \textbf{定时器管理}:在登出时通过 \texttt{clearInterval(timeInterval)} 清理定时器,防止内存泄漏。
    \end{itemize}
    
    \item \textbf{流量统计功能}:实现了完整的流量跟踪系统,包括:
    \begin{itemize}
        \item \textbf{流量数据加载}:从localStorage中读取用户历史流量数据,格式为 \texttt{\{ "用户名": 流量值 \}},如果是首次登录则初始化为0GB。
        \item \textbf{流量动态更新}:通过 \texttt{setInterval(computeTraffic, 1000)} 每秒调用一次流量计算函数,每次增加3.33GB,最大限制为50GB。
        \item \textbf{流量数据存储}:使用 \texttt{localStorage.setItem('trafficUsage', JSON.stringify(traffic))} 将流量数据持久化存储。
        \item \textbf{流量显示格式}:使用 \texttt{trafficUsage.toFixed(2) + ' GB'} 确保流量显示保留两位小数。
    \end{itemize}
    
    \item \textbf{进度条动态更新}:通过 \texttt{updateProgressBar()} 函数实现,根据当前流量使用量计算百分比(流量使用量/50GB*100),并通过 \texttt{progressFill.style.width} 动态设置进度条宽度。
    
    \item \textbf{登出功能}:监听登出按钮点击事件,清除所有定时器防止内存泄漏,使用 \texttt{localStorage.removeItem('currentUsername')} 清除当前用户信息但保留流量数据,并通过 \texttt{window.location.href = '../index.html'} 返回登录页面。
\end{itemize}

\textbf{数据传递与存储机制}:使用浏览器的 \texttt{localStorage} API 实现数据的持久化存储,包括用户注册信息、当前登录状态和流量使用数据,确保页面间的数据一致性和用户数据的持久性。

\section{使用说明}

将文件下载后,可以发现/src/index.html。可以直接用浏览器打开。此外/src/csstable为样式表,/src/hw1为第一次作业所做页面,这里对于部分内容做了更改后拿来使用,src/source为图片资源。整体上,对于/src/hw1/index\_success.html和/src/index.html进行了部分修改,以实现本次作业目标。

\section{问题及解决办法}

在实现过程中遇到的主要问题包括:

\begin{itemize}
    \item \textbf{流量统计逻辑错误}:最初使用基于总秒数的计算方式 \texttt{seconds * 3.33},导致流量一次性增加过多。后改为使用定时器每秒增加固定值 \texttt{trafficUsage += 3.33},确保流量按预期增长。
    
    \item \textbf{流量数据持久化}:最初在登出时删除所有流量数据 \texttt{localStorage.removeItem('trafficUsage')},导致用户数据丢失。后改为保留流量数据,只删除当前用户名,确保用户流量使用历史的连续性。
            
    \item \textbf{用户名密码储存安全性}:最初考虑使用简单的字符串存储用户信息,后发现使用JSON格式更便于管理和扩展。为此我学习了JSON相关的内容,通过 \texttt{JSON.stringify()} 和 \texttt{JSON.parse()} 实现数据的序列化和反序列化,确保数据结构的完整性。
\end{itemize}

\section{参考资料}

除了前两次作业中的参考内容之外,本次作业中,为了实现流量统计功能,参考了JavaScript定时器API文档、localStorage数据存储最佳实践,以及JSON数据格式处理的相关资料。
此外使用了大模型进行代码注释的补全,以及代码格式的规范,确保代码的可读性和可维护性。

\end{document}